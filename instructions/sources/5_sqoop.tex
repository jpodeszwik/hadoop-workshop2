\documentclass[11pt]{article}
\usepackage[utf8]{inputenc}
\usepackage{polski}
\usepackage[polish]{babel}

\begin{document}
\pagenumbering{gobble}
\section*{Sqoop}

Ściągnięcie danych z bazy mysql na hdfs:
\begin{enumerate}
\item zainstaluj na vm-cluster-node1 serwer mysql 'sudo apt-get install mysql-server'
\item Skopiuj plik 'dump.sql' na vm-cluster-node1 i wykonaj polecenie 'mysql -u root \textless dump.sql'
\item w pliku '/etc/mysql/my.cnf ustaw 'bind-address = 0.0.0.0'
\item zrestartuj mysql 'sudo /etc/init.d/mysql restart'
\item ściągnij connector dla mysqla http://dev.mysql.com/get/Downloads/Connector-J/mysql-connector-java-5.1.39.tar.gz i rozpakuj (tar -xzf mysql-connector-java-5.1.39.tar.gz)
\item W rozpakowanym archiwum znajdź mysql-connector-java-5.1.39-bin.jar i wrzuć na vm-cluster-node2 do katalogu '/var/lib/sqoop'
\item Na vm-cluster-node2 uruchom import tabeli sqoop.ip\_name za pomocą komendy: 'sqoop import --connect jdbc:mysql://vm-cluster-node1:3306/sqoop --username sqoop --password sqoop\_pwd --table ip\_name -m 1 --target-dir /user/vagrant/ip\_name
\end{enumerate}

\end{document}
