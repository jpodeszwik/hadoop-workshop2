\documentclass[11pt]{article}
\usepackage[utf8]{inputenc}
\usepackage{polski}
\usepackage[polish]{babel}

\begin{document}
\pagenumbering{gobble}
\section*{Instalacja}

\begin{enumerate}
\item Zainstaluj pakiet vagrant. Jeśli korzystasz z ubuntu, to możesz to zrobić za pomocą polecenia 'sudo apt-get install vagrant'
\item Wykonaj polecenie 'vagrant plugin install vagrant-hostmanager'
\item Sklonuj repozytorium https://github.com/jpodeszwik/virtual-hadoop-cluster
\item Korzystając z terminala wejdź do katalogu virtual-hadoop-cluster
\item wpisz polecenie ‘vagrant up’ i poczekaj aż się wykona. Możesz zostać poproszony o podanie hasła do sudo.
\item wejdź w przeglądarce na adres http://vm-cluster-node1:7180 i zaloguj się admin / admin
\item Przeczytaj licencję ;) zaakceptuj ją zaznaczając checkbox i kliknij 'Continue'
\item Wybierz edycję ‘cloudera express’ i kliknij 'Continue'
\item kolejne okno pomiń klikająć 'Continue'
\item Na kolejnym ekranie należy wpisać nazwy hostów z których będzie składał się klaster. Wpisz ‘vm-cluster-node[1-4]’, kliknij search, poczekaj chwilę i w liście pod spodem powinny pojawić się 4 hosty. Następnie kliknij continue.
\item Kliknij continue.
\item Przeczytaj licencję ;) zaakceptuj ją zaznaczając checkbox. Pojawi się dodatkowy checkbox, \textbf{którego nie zaznaczaj}. Następnie kliknij Continue.
\item Kolejny ekran pomiń klikając continue.
\item Dla ustawienia ‘Login To All Hosts As’ wybierz ‘Another user’. W polu, które pojawi się obok wpisz ‘vagrant’. Niżej w polach ‘Enter Password’ i ‘Confirm Password’ wpisz ‘vagrant’ i kliknij Continue.
\item Pojawi się okno instalacji pakietów cloudera managera na hostach. Po zakończeniu instalacji kliknij Continue.
\item Pojawi się okno ściągania paczki z hadooem. Po zakończeniu procesu kliknij Continue.
\item Uruchomi się okno host inspectora. Kliknij ‘Finish’
\item Wybierz opcję ‘Core Hadoop’ i kliknij ‘Continue’
\item Na kolejnym ekranie można zmienić rozkład ról w klastrze. Zostaw domyślny i kliknij ‘Continue’
\item Na kolejnym ekranie można ustawić bazy danych. Zostaw domyślne, kliknij ‘Test Connection’ i następnie ‘Continue’
\item Pojawi się ekran z podstawowymi ustawieniami klastra. Ustaw wartość HDFS Block Size  na 16 MiB i kliknij continue.
\item Powinien rozpocząć się proces pierwszego uruchomienia klastra. Może on potrwać kilka minut. Po zakończeniu kliknij 'Continue'
\item Kliknij 'Finish' aby zakońćzyć instalację. Zostaniesz przeniesiony do widoku klastra.
\end{enumerate}

\end{document}
