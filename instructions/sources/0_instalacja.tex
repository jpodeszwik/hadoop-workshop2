\documentclass[11pt]{article}
\usepackage[utf8]{inputenc}
\usepackage{polski}
\usepackage[polish]{babel}

\begin{document}

\begin{enumerate}
\item Korzystając z terminala wejdź do katalogu virtual-hadoop-cluster
\item wpisz polecenie ‘vagrant up’ i poczekaj aż się wykona. Możesz zostać poproszony o podanie hasła do sudo. Jeśli tak się stanie, wpisz ‘infoshare’.
\item wejdź w przeglądarce na adres http://vm-cluster-node1:7180 i zaloguj się admin / admin
\item Przeczytaj licencję ;) zaakceptuj ją zaznaczając checkbox i kliknij 'Continue'
\item Wybierz edycję ‘cloudera express’ i kliknij 'Continue'
\item kolejne okno pomiń klikająć 'Continue'
\item Na kolejnym ekranie należy wpisać nazwy hostów z których będzie składał się klaster. Wpisz ‘vm-cluster-node[1-4]’, kliknij search, poczekaj chwilę i w liście pod spodem powinny pojawić się 4 hosty. Następnie kliknij continue.
\item Na kolejnym ekranie wybiera się sposób dostarczania paczek. Na potrzeby warsztatu na maszynie vm-cluster-node1 zostało utworzone repozytorium paczek z którego skorzystamy, żeby nie obciążać łącza. Przy opcji ‘Use Parcels` kliknij w przycisk ‘More Options’.
\item Usuń wszystkie adresy z ‘Remote Parcel Repository URLs’. Zamiast nich dodaj adres ‘http://vm-cluster-node1/parcels/’ i kliknij ‘Save changes’
\item Widok sposobu dostarczania paczek powinien się odświeżyć. W sekcji ‘Select the version of CDH` powinna zostać 1 zaznaczona opcja ‘CDH-5.9.0-1.cdh5.9.0.p0.23’. Kliknij continue.
\item Przeczytaj licencję ;) zaakceptuj ją zaznaczając checkbox. Pojawi się dodatkowy checkbox, \textbf{którego nie zaznaczaj}. Następnie kliknij Continue.
\item Kolejny ekran pomiń klikając continue.
\item Dla ustawienia ‘Login To All Hosts As’ wybierz ‘Another user’. W polu, które pojawi się obok wpisz ‘vagrant’. Niżej w polach ‘Enter Password’ i ‘Confirm Password’ wpisz ‘vagrant’ i kliknij Continue.
\item Pojawi się okno instalacji pakietów cloudera managera na hostach. Aby zaoszczędzić czas, na maszynach warsztatowych te pakiety zostały już poinstalowane, więc instalacja powinna przejść szybko. Po zakończeniu instalacji kliknij Continue.
\item Pojawi się okno ściągania paczki z hadooem. Na warsztacie korzystamy z lokalnego repozytorium pakietów, więc paczka powinna ściągnąć się szybko. Po zakończeniu procesu kliknij Continue.
\item Uruchomi się okno host inspectora. Kliknij ‘Finish’
\item Wybierz opcję ‘Core Hadoop’ i kliknij ‘Continue’
\item Na kolejnym ekranie można zmienić rozkład ról w klastrze. Zostaw domyślny i kliknij ‘Continue’
\item Na kolejnym ekranie można ustawić bazy danych. Zostaw domyślne, kliknij ‘Test Connection’ i następnie ‘Continue’
\item Pojawi się ekran z podstawowymi ustawieniami klastra. Ustaw wartość HDFS Block Size  na 16 MiB i kliknij continue.
\item Powinien rozpocząć się proces pierwszego uruchomienia klastra. Może on potrwać kilka minut. Po zakończeniu kliknij 'Continue'
\item Kliknij 'Finish' aby zakońćzyć instalację. Zostaniesz przeniesiony do widoku klastra.
\end{enumerate}

\end{document}
