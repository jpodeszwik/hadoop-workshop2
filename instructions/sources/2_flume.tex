\documentclass[11pt]{article}
\usepackage[utf8]{inputenc}
\usepackage{polski}
\usepackage[polish]{babel}
\usepackage{listings}
\lstset{
  breaklines=true,
   frame=single,
   showstringspaces=false
}
\begin{document}

Instalacja flume:
\begin{enumerate}
\item Wejdź na interfejs webowy cluodera managera http://vm-cluster-node1:7180
\item Kliknij strzałkę w dół po prawej stronie napisu 'Cluster 1', a następnie 'Add Service'
\item Wybierz z listy 'Flume' i kliknij continue
\item kliknij na 'Select hosts'. Zaznacz checkbox przy 'vm-cluster-node2', a następnie kliknij 'OK' i 'Continue'.
\item kliknij 'Finish'
\item kliknij na nowo powstały serwis 'Flume', a następnie wejdź w zakładkę 'Configuration'
\item Przekopiuj konfigurację z pliku 'flume.conf' do pola 'Configuration File'
\item Dodaj do 'Plugin directories' wartość '/home/vagrant/flume.plugins.d'
\item Kliknij 'save changes'
\item Utwórz na vm-cluster-node2 katalog '/home/vagrant/flume.plugins.d/interceptors/lib' i wrzuć do niego zbudowanego jara z interceptorami.
\item Kliknij u góry na 'Actions'
\item Kliknij akcję 'Start'
\item Wejdź na https://github.com/jpodeszwik/Fake-Apache-Log-Generator i sklonuj repozytorium
\item Postępuj zgodnie z instrukcją instalacji w README
\item uruchom generator spreparowanych logów apache'a za pomocą komendy './run.sh vm-cluster-node2 9999'
\item wylistuj katalog na hdfsie /user/flume/events i zobacz czy pojawiają się logi apache'a
\end{enumerate}

\pagebreak

Zadania:
\begin{enumerate}
\item Napisz interceptor, który odkoduje datę z linijki loga i ustawi odpowiednią datę w nagłówku. Skorzystaj z nagłówka ‘timestamp’ i umieść w nim unixowy timestamp wyznaczony na podstawie daty odkodowanej z eventu. Eventy z przedwczorajszą datą wygeneruj za pomocą polecenia './run2.sh vm-cluster-node2 9999'
\item Skorzystaj z możliwości multiplexacji, żeby odfiltrować błędne linijki loga. Przekieruj błędne linijki do innego katalogu. Błędne eventy wygeneruj za pomocą komendy './run3.sh vm-cluster-node2 9999'

https://flume.apache.org/FlumeUserGuide.html\#multiplexing-the-flow
\end{enumerate}

\end{document}
