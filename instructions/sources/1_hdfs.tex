\documentclass[11pt]{article}
\usepackage[utf8]{inputenc}
\usepackage{polski}
\usepackage[polish]{babel}

\begin{document}

Przydatne komendy hdfs:
\begin{enumerate}
\item ls - listuje katalog

hdfs dfs -ls \textit{ścieżka}

\item cat - wypisuje zawartość pliku

hdfs dfs -cat \textit{ścieżka}

\item mkdir - tworzy katalog

hdfs dfs -mkdir \textit{ścieżka}
 
\item rm - usuwa plik

hdfs dfs -rm \textit{ścieżka}

\item chmod - zmienia prawa dostępu do pliku

hdfs dfs -chmod \textit{prawa\_dostępu} \textit{ścieżka}

\item chown - zmienia właściciela pliku

hdfs dfs -chown \textit{user}:\textit{grupa} \textit{ścieżka}

\item cp - kopiuje plik na hdfsie

hdfs dfs -cp \textit{źródło} \textit{cel}

\item mv - przenosi plik na hdfsie

hdfs dfs -mv \textit{źródło} \textit{cel}

\item put - umieszka lokalny plik na hdfsie

hdfs dfs -put \textit{ścieżka\_lokalna} \textit{ścieżka\_na\_hdfs}

\item get - pobiera plik z hdfsa

hdfs dfs -get \textit{ścieżka\_na\_hdfs} \textit{ścieżka\_lokalna}

\item touchz - tworzy pusty plik

hdfs dfs -touchz \textit{ścieżka}

\item appendToFile - przepisuje plik do końca pliku na hdfsie

hdfs dfs -appendToFile \textit{ścieżka\_lokalna} \textit{ścieżka\_na\_hdfs}

\end{enumerate}

Jeżeli klaster nie jest zabezpieczony kerberosem, to możemy wykorzystać zmienną środowiskową HADOOP\_USER\_NAME, żeby ustawić użytkownika z którego wykonane zostanie polecenie, np.\\*
HADOOP\_USER\_NAME hdfs dfs -mkdir /katalog
\\*
\\*
\\*
Zadania:
\begin{enumerate}
\item Utwórz katalogi:\\*
vagrant:vagrant /user/vagrant
\\*
vagrant:vagrant /user/vagrant/inputs
\\*
vagrant:vagrant /user/vagrant/outputs
\\*
flume:flume /user/flume
\\*
kursant:kursant /user/kursant
\item wrzuć pliki loremipsum i apache\_logs do /user/vagrant/inputs
\item zmień prawa do odczytu do katalogu /user/kursant tak, żeby tylko kursant mógł go odczytać
\item spróbuj odczytać katalog jako user ’vagrant’
\item spróbuj odczytać katalog jako user ’hdfs’
\end{enumerate}

\end{document}
